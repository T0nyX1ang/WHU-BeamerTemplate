%!TeX Options=--shell-escape

% demo.tex v1.3.0
% A beamer theme based on the spots in Wuhan University.
% Author: Tony Xiang
% Licensed under the LaTeX Project Public License, v1.3c or later.

\documentclass{beamer}
\usepackage{ctex}
\usepackage{tikz}
\usepackage{hologo}
\usepackage{cncolours}
\usepackage{amsthm,amsmath}
\usepackage{fancyvrb}
\usepackage{url}
\usepackage{minted}
\usepackage{verbatim}

\usetikzlibrary{calc}
\usetheme{WHUSpot}
\usefonttheme{serif}
\usecolorstyle{default}
\mode<presentation>

\title{武汉大学主题 Beamer 模板 - WHUSpot}
\author{向飞宇}
\institute{武汉大学}
\date{\today}
\Plogo{./logos/whulogoblack.eps}

\begin{document}

\begin{frame}
    \titlepage
\end{frame}

\begin{frame}
    \frametitle{目录}
    \tableofcontents
\end{frame}

\section{使用方法}

\subsection{subsection没有特别的页面,但是会写入目录中}

\subsubsection{subsubsection也一样}

\begin{frame}[fragile]
    \frametitle{使用方法与注意事项}
    \begin{itemize}
        \item 本模板适用于\alert{较短的,内容层级不深的}展示.
        \item 用~\verb|\usetheme{WHUSpot}| 即可载入模板,当然也需要将 beamerthemeWHUSpot.sty 放在合适的地方(其实在同目录下就行).
        \item 请使用 \hologo{XeLaTeX} 编译,一般而言使用\alert{TeX Live 2018}及以上版本能够保证正常编译(当然稍微低版本的TeX Live应该也没问题).
        \item 本人\alert{没有设计subsection及以下层次}的页面,仅在有section时调用一个特别的页面,请谨慎组织标题内容.
    \end{itemize}
\end{frame}

\begin{frame}
    \frametitle{列表与枚举环境}
    列表和枚举环境均可以定义三级,但是本模板中\alert{建议使用列表环境},以下为列表环境和枚举环境示例.
    \begin{itemize}
        \item 这是一级内容. \pause
        \item 这是一级的另一个内容. \pause
        \begin{itemize}
            \item 这是二级内容. \pause
            \item 这是二级的另一个内容. \pause
            \begin{itemize}
                 \item 这是三级内容. \pause
                 \item 没有四级内容的环境了. \pause
            \end{itemize} 
        \end{itemize}
    \end{itemize}

    \begin{enumerate}
        \item This is an enumeration. \pause
        \item This is another one. \pause
        \begin{enumerate}
             \item This is an enumeration. \pause
             \item Stop here. 
         \end{enumerate} 
    \end{enumerate}
\end{frame}

\begin{frame}
    \frametitle{普通,强调,示例块与分栏}
    黑色的是普通文字.红色(酡红)的是\alert{强调}文字. \pause

    以下是普通块,强调块和示例块(主要就是上面线的颜色不一样),以及它们的分栏显示结果,注意\alert{最多只分两栏},否则视觉效果不好.
    \begin{columns}
        \begin{column}{4cm}
            \begin{block}{普通块}
                这是一个普通块. \pause
            \end{block}

            \begin{alertblock}{强调块}
                这是一个强调块. \pause
            \end{alertblock}

            \begin{exampleblock}{示例块}
                这是一个示例块. \pause
            \end{exampleblock}
        \end{column}

        \begin{column}{4cm}
            \begin{block}{普通块}
                这是一个普通块. \pause
            \end{block}

            \begin{alertblock}{强调块}
                这是一个强调块. \pause
            \end{alertblock}

            \begin{exampleblock}{示例块}
                这是一个示例块. \pause
            \end{exampleblock}
        \end{column}
    \end{columns}

\end{frame}

\begin{frame}
    \frametitle{定理与脚注}
    \begin{theorem}[一元二次方程的解]
    \begin{equation*}
        a x^2 + b x + c = 0, a \neq 0 \mbox{~的解为~} x_{1, 2} = \frac{-b \pm \sqrt{b^2 - 4 a c}}{2 a} 
    \end{equation*}
    \end{theorem}
    \begin{proof}
        用配方法,可得:
        \begin{equation*}
            (x + \frac{b}{2a})^2 = \frac{b^2 - 4 a c}{4a^2}
        \end{equation*}
        开根号\footnote{需要在复数意义下考虑.}即可.
    \end{proof}
\end{frame}

\begin{frame}[fragile]
    \frametitle{代码显示}
    代码显示可以结合~\verb|minted| 宏包或者其它用于显示代码的宏包,但是一般需要每个frame中传入fragile参数.在使用~\verb|minted| 宏包时,编译时需要加上~\verb|--shell-escape| 参数,并用~\verb|Python| 安装~\verb|Pygments| 依赖.以下是一个示例.
    \begin{minted}{c}
    #include<stdio.h>
    int main(int argc, char const *argv[])
    {
        printf("%s\n", "hello world!");
        return 0;
    }
    \end{minted}
\end{frame}

\begin{frame}[fragile]
    \frametitle{自定义}
    \begin{itemize}
        \item 配色:目前包括两套配色方案~\verb|default| 和~\verb|dark| ,详见~\verb|colors| 文件夹,自定义配色方案需要给出相应参数的定义,引用配色方案可以使用~\verb|\usecolorstyle{...}| 命令.
        \item 绘图模型:目前包括四个模型,详见~\verb|models| 文件夹,引用模型需要传入相应参数,并置于~\verb|tikzpicture| 环境中.
        \item Logo:默认为空,一般建议使用新定义的~\verb|Plogo{filepath}|命令插入logo,~\verb|filepath|代表logo的路径.
    \end{itemize}
\end{frame}

\section{设计灵感}

\begin{frame}
    \frametitle{一瓣瓣的樱花}
    \begin{figure}
    \centering
    \begin{tikzpicture}[scale=.8]
        \foreach \x in {1,...,5}            
        {
            \pause
            \drawpetal{(\x - 1) * 72}{1}
        }
    \end{tikzpicture}
    \end{figure}
\end{frame}

\begin{frame}
    \frametitle{老图书馆不是一日建成的}
    \begin{figure}
    \centering
    \def\bdcolors{蟹壳青}
    \begin{tikzpicture}[scale=.4]
        \useasboundingbox (-0.2, 0.5) rectangle (0.05, 10);
        \fill[\bdcolors] (-1, 0) rectangle (1, 2);
        \draw[\bdcolors] (-1.2, 0) rectangle (1.2, 2.2);
        \draw[line width=2.5pt,\bdcolors] (-6, 0) -- (6, 0);
        \foreach \x in {1,...,5}
        {
            \draw[line width=1.5pt, \bdcolors] (-4.5 - \x * 0.2, -\x / 3) -- (4.5 + \x * 0.2, -\x / 3);
        }
        \fill[\bdcolors] (-12, -2) -- (12, -2) -- (11.5, -2.2) -- (-11.5, -2.2) -- cycle;
        \fill[\bdcolors] (-11.3, -2.28) -- (11.3, -2.28) -- (10.8, -2.48) -- (-10.8, -2.48) -- cycle;
        \fill[\bdcolors] (-6, -2.56) -- (6, -2.56) -- (5.5, -2.76) -- (-5.5, -2.76) -- cycle;
        \pause

        \foreach \i in {-1, 1} 
            \filldraw[\bdcolors] (-5.8 * \i, 0) -- (-5.8 * \i, -0.3) -- (-5.5 * \i, -0.3) -- (-6.4 * \i, -1.8) -- (-11.4 * \i, -1.8) -- (-10.68 * \i, -0.6) -- (-10.5 * \i, -0.6) -- (-10.5 * \i, 3.5) -- (-6 * \i, 3.5) -- (-6 * \i, 0) -- cycle;
        \foreach \j in {8.97, 7.72, 6.47}
            \filldraw[rounded corners, color=backgroundcolor] (\j, 2.4) rectangle (\j+ 1, 3);
        \foreach \j in {-8.97, -7.72, -6.47}
            \filldraw[rounded corners, color=backgroundcolor] (\j, 2.4) rectangle (\j- 1, 3);
        \pause

        \filldraw[\bdcolors] (-10.5, 3.9) rectangle (10.5, 4.6);
        \filldraw[\bdcolors] (-10.5, 3.5) rectangle (10.5, 3.7);
        \foreach \k in {-5.6, -3.1, 3.5}
            \filldraw[\bdcolors] (\k, 0) -- (\k, 3.9) -- (\k + 0.5, 3.9) -- (\k + 0.5, 3.5) -- (\k + 0.2, 3.3) -- (\k + 0.2, 0) -- cycle;
        \foreach \k in {5.6, 3.1, -3.5}
            \filldraw[\bdcolors] (\k, 0) -- (\k, 3.9) -- (\k - 0.5, 3.9) -- (\k - 0.5, 3.5) -- (\k - 0.2, 3.3) -- (\k - 0.2, 0) -- cycle;
        \filldraw[\bdcolors] (-0.25, 3.7) rectangle (0.25, 4);
        \pause

        \filldraw[\bdcolors] (-10.5, 4.8) [out=165,in=-78]to (-12.8, 7) [out=-75,in=-180]to (-10.8, 6) [out=0,in=-95]to (-8.9, 8) [out=-85,in=-180]to (-6.8, 5.9) [out=0,in=-120]to (-5, 6.8) [out=-95,in=15]to (-6, 5.0) -- (-6, 4.8) -- cycle; 
        \filldraw[\bdcolors] (10.5, 4.8) [out=15,in=-102]to (12.8, 7) [out=-105,in=0]to (10.8, 6) [out=180,in=-85]to (8.9, 8) [out=-95,in=0]to (6.8, 5.9) [out=180,in=-60]to (5, 6.8) [out=-85,in=165]to (6, 5.0) -- (6, 4.8) -- cycle;
        \filldraw[\bdcolors] (-6, 4.8) rectangle (6, 5.0);
        \foreach \i in {-1, 1}
        {
            \filldraw[\bdcolors] (-8.4 * \i, 6.9) -- (-5.1 * \i, 6.9) -- (-5.3 * \i, 6.6) -- (-8.1 * \i, 6.6) -- cycle;
            \filldraw[\bdcolors] (-7.99 * \i, 6.45) -- (-5.4 * \i, 6.45) -- (-5.52 * \i, 6.35) -- (-7.87 * \i, 6.35) -- cycle;     
        }
        \filldraw[\bdcolors] (-4.88, 6.9) -- (4.88, 6.9) -- (4.9, 6.6) -- (-4.9, 6.6) -- cycle;
        \filldraw[\bdcolors] (-4.93, 6.45) -- (4.93, 6.45) -- (4.95, 6.35) -- (-4.95, 6.35) -- cycle;
        \pause

        \foreach \i in {-1, 1}
        {
            \filldraw[\bdcolors] (-5.5 * \i, 5.1) -- (-5.5 * \i, 5.2) -- (-5.3 * \i, 5.3) -- (-5.3 * \i, 5.1) -- cycle;
            \filldraw[\bdcolors] (-5.2 * \i, 5.1) -- (-5.2 * \i, 5.35) -- (-5.05 * \i, 5.85) -- (-5.05 * \i, 5.1) -- cycle;
        }
        \foreach \i in {0,...,24}
            \filldraw[\bdcolors] (-4.9 + \i * 0.4, 5.1) rectangle (-4.9 + \i * 0.4 + 0.2, 6.25);
        \pause

        \filldraw[\bdcolors] (-5, 7.7) [out=180,in=-90]to (-6.2, 8.5) [out=-30,in=-150]to (-4.4, 8.5) -- (-4.5, 9.6) to (-3.5, 8.8) -- (-2.7, 9.1) -- (-2.7, 10.2) -- (2.7, 10.2) -- (2.7, 9.1) -- (3.5, 8.8) to (4.5, 9.6) -- (4.4, 8.5) [out=-30,in=-150]to (6.2, 8.5) [out=-90,in=0]to (5, 7.7) -- (3.1, 8.2) -- (-3.1, 8.2) -- cycle;
        \filldraw[rounded corners=1pt,\bdcolors] (-2.9, 10.0) rectangle (-2.5, 10.4);
        \filldraw[rounded corners=1pt,\bdcolors] (2.9, 10.0) rectangle (2.5, 10.4);
        \foreach \i in {-1, 1}
            \filldraw[rounded corners=3pt,rotate around={13 * \i:(-4.7 * \i,7.1)},\bdcolors] (-4.7 * \i, 7.1) rectangle (-3.1 * \i, 7.5); 
        \foreach \i in {0,...,3}
            \filldraw[rounded corners=3pt, \bdcolors] (-2.95 + \i * 1.5, 7.5) rectangle (-1.55 + \i * 1.5, 7.9);
        \pause

        \filldraw[\bdcolors] (-0.3, 10.2) rectangle (0.3, 10.3);
        \filldraw[\bdcolors] (-0.3, 10.4) [bend left=10]to (-0.3, 11.2) -- (-0.2, 11.3) -- (-0.3, 11.4) -- (-0.3, 11.6) -- (0.3, 11.6) -- (0.3, 11.4) -- (0.2, 11.3) -- (0.3, 11.2) [bend left=10]to (0.3, 10.4) -- cycle;
        \filldraw[\bdcolors] (-0.2, 11.7) rectangle (0.2, 11.8);
    \end{tikzpicture}
    \end{figure}
\end{frame}

\begin{frame}
    \frametitle{鸣谢}
    \begin{itemize}
        \item 武汉大学.
        \item 用cncolours当调色板. \\ \url{https://github.com/liantze/pgfornament-han}
        \item progress bar的思路. \\ \url{https://github.com/DjalelBBZ/Algiers-beamer-template} 
    \end{itemize}
\end{frame}

\section{联系,bug反馈,授权}
\begin{frame}
    \frametitle{联系,bug反馈,授权}
    \begin{itemize}
        \item 最新版本在 \\ \url{https://github.com/T0nyX1ang/WHU-BeamerTemplate} 上发布.
        \item 可以直接在Github上发issue提出bug反馈与功能建议.
        \item 使用LPPL-1.3c(及以后版本)授权.
    \end{itemize}
\end{frame}

\end{document}