\documentclass{beamer}
\usepackage{tikz}
\usepackage{cncolours}
\usetikzlibrary{calc}
\usetheme{WHUSpot}
\mode<presentation>

\title{WHUSpot Beamer Template}
\author{Tony Xiang}
\institute{Wuhan University}

\begin{document}
\begin{frame}
	\frametitle{\\}
	\titlepage
\end{frame}

\section{Euclid's Talk}
\begin{frame}
    \frametitle{What Are Prime Numbers?}
    \begin{definition}
        A \alert{prime number} is a number that has exactly two divisors.
    \end{definition}
    \begin{itemize}
        \item 2 is prime (two divisors: 1 and 2).
        \pause
        \item 3 is prime (two divisors: 1 and 3).
        \pause
        \item 4 is not prime (\alert{three} divisors: 1, 2 and 4) 
    \end{itemize}
\end{frame}

\begin{frame} % [t] vertically centered
    \frametitle{There Is No Largest Prime Number}
    \framesubtitle{The proof uses \textit{reductio ad absurdum}.}
    \begin{theorem}
        There is no largest prime number.
    \end{theorem}
    \begin{proof}
        \begin{enumerate}
            \item<1-> Suppose $p$ were the largest prime number.
            \item<2-> Let $q$ be the product of the first $p$ numbers.
            \item<3-> Then $q + 1$ is not divisible by any of them.
            \item<1-> But $q + 1$ is greater than $1$, thus divisible by some prime number not in the first $p$ numbers. \qedhere
        \end{enumerate}
    \end{proof}
    \uncover<4->{The proof used \textit{reductio ad absurdum}.}
\end{frame}

\begin{frame}
    \frametitle{What's Still To Do?}
    \begin{block}{Answered Questions}
        How many primes are there?
    \end{block}
    \begin{block}{Open Questions}
        Is every even number the sum of two primes?
    \end{block}
\end{frame}

\begin{frame}
    \frametitle{What’s Still To Do?} % nested itemize
        \begin{itemize}
        \item Answered Questions
            \begin{itemize}
                \item How many primes are there?
            \end{itemize}
        \item Open Questions
            \begin{itemize}
                \item Is every even number the sum of two primes?
            \end{itemize}
    \end{itemize}
\end{frame}

\begin{frame}
    \frametitle{What's Still To Do?} % more sophisticated, columns
    \begin{columns}
        \column{.5\textwidth}
        \begin{block}{Answered Questions}
            How many primes are there?
        \end{block}     
        \column{.5\textwidth}
        \begin{block}{Open Questions}
            Is every even number the sum of two primes?
            \cite{Goldbach1742}
        \end{block}
    \end{columns}
\end{frame}

\begin{frame}[fragile]
\frametitle{An Algorithm For Finding Primes Numbers.} % more verbatim
    \begin{semiverbatim}
        \uncover<1->{\alert<0>{int main (void)}}
        \uncover<1->{\alert<0>{\{}}
        \uncover<1->{\alert<1>{ \alert<4>{std::}vector<bool> is_prime (100, true);}}
        \uncover<1->{\alert<1>{ for (int i = 2; i < 100; i++)}}
        \uncover<2->{\alert<2>{ if (is_prime[i])}}
        \uncover<2->{\alert<0>{ \{}}
        \uncover<3->{\alert<3>{ \alert<4>{std::}cout << i << " ";}}
        \uncover<3->{\alert<3>{ for (int j = i; j < 100;}}
        \uncover<3->{\alert<3>{ is_prime [j] = false, j+=i);}}
        \uncover<2->{\alert<0>{ \}}}
        \uncover<1->{\alert<0>{ return 0;}}
        \uncover<1->{\alert<0>{\}}}
    \end{semiverbatim}
    \visible<4->{Note the use of \alert{\texttt{std::}}.}
\end{frame}

\section{Some interesting examples}
\begin{frame}[<+->]
    \frametitle{<+-> on a frame}
    \begin{theorem}
    $A = B$.
    \end{theorem}
    \begin{proof}
    \begin{itemize}
    \item Clearly, $A = C$.
    \item As shown earlier, $C = B$.
    \item<3-> Thus $A = B$.
    \end{itemize}
    \end{proof}
\end{frame}

\begin{frame}
\frametitle{Overlays}
\begin{itemize}
\item
Shown from first slide on.
\pause
\item
Shown from second slide on.
\begin{itemize}
\item
Shown from second slide on.
\pause
\item
Shown from third slide on.
\end{itemize}
\item
Shown from third slide on.
\pause
\item
Shown from fourth slide on.
\end{itemize}
Shown from fourth slide on.
\begin{itemize}
\onslide
\item
Shown from first slide on.
\pause
\item
Shown from fifth slide on.
\end{itemize}
\end{frame}

\part{Review of Previous Lecture}
\frame{\partpage}

\begin{frame}
    \begin{itemize}
        \item<1-> First item.
        \item<2-> Second item.
        \item<3-> Third item. 
    \end{itemize}
    \hyperlink{jumptosecond}{\beamergotobutton{Jump to second slide}}
    \hypertarget<2>{jumptosecond}{}
\end{frame}

\frame<1-2>[label=myframe]
{
\frametitle{repeating a frame}
\begin{itemize}
    \item<alert@1> First subject.
    \item<alert@2> Second subject.
    \item<alert@3> Third subject.
\end{itemize}
}
\frame
{
    Some stuff explaining more on the second matter.
}
\againframe<3>{myframe}

\begin{frame}
    \begin{itemize}
    \item<1-> Eggs
    \item<2-> Plants
    \note[item]<2>{Tell joke about plants.}
    \note[item]<2>{Make it short.}
    \item<3-> Animals
    \end{itemize}
\end{frame}

\begin{thebibliography}{10}
    \bibitem{Goldbach1742}[Goldbach, 1742]
    Christian Goldbach.
    \newblock A problem we should try to solve before the ISPN ’43 deadline,
    \newblock \emph{Letter to Leonhard Euler}, 1742.
\end{thebibliography}

\end{document}