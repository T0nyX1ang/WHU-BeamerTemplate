\documentclass{beamer}
\usepackage{ctex}
\usepackage{tikz}
\usepackage{hologo}
\usepackage{cncolours}
\usepackage{amsthm,amsmath}
\usepackage{fancyvrb}
\usepackage{url}

\usetikzlibrary{calc}
\usetheme{WHUSpot}
\usefonttheme{serif}
\mode<presentation>

\title{武汉大学主题 Beamer 模板 - WHUSpot}
\author{向飞宇}
\institute{武汉大学}
\date{\today}

\begin{document}

\begin{frame}
    \frametitle{\\}
    \titlepage
\end{frame}

\section{使用方法}

\begin{frame}
    \frametitle{使用方法与注意事项}
    \begin{itemize}
        \item 用 $\backslash \mbox{usetheme\{WHUSpot\}}$ 即可载入模板,当然也需要将 beamerthemeWHUSpot.sty 放在合适的地方(其实在同目录下就行).
        \item 请使用 \hologo{XeLaTeX} 编译,一般而言使用\alert{TeX Live 2018}及以上版本能够保证正常编译(当然稍微低版本的TeX Live应该也没问题).
        \item 本人\alert{没有设计subsection及以下层次}的页面,仅在有section时调用一个特别的页面,请谨慎组织标题内容.
        \item 请不要轻易改变配色方案.
    \end{itemize}
\end{frame}

\begin{frame}
    \frametitle{列表与枚举环境}
    \begin{itemize}
        \item 列表环境是beamer默认的,可以修改. \pause
        \item 支持多级列表,但修改环境要逐级修改,\alert{不建议}使用. \pause
    \end{itemize}

    \begin{enumerate}
        \item 枚举环境是beamer默认的,可以修改. \pause
        \item 支持多级枚举,但修改环境要逐级修改,\alert{不建议}使用.
    \end{enumerate}
\end{frame}

\section{设计灵感}
\begin{frame}
    \frametitle{配色}
    红,蓝,黑为主,绿,很浅的白为辅.

    用cncolours包配色.
\end{frame}

\begin{frame}
    \frametitle{鸣谢}
    \begin{itemize}
        \item 武汉大学.
        \item 用cncolours当调色板. \\ \url{https://github.com/liantze/pgfornament-han}
        \item progress bar的思路. \\ \url{https://github.com/DjalelBBZ/Algiers-beamer-template} 
    \end{itemize}
\end{frame}

\section{联系,bug反馈,授权}
\begin{frame}
    \frametitle{联系,bug反馈,授权}
    \begin{itemize}
        \item 最新版本在 \\ \url{https://github.com/T0nyX1ang/WHU-BeamerTemplate} 上发布.
        \item bug反馈可以直接在Github上发issue.
        \item 使用LPPL-1.3c(及以后版本)授权.
    \end{itemize}
\end{frame}

\begin{frame}
    \frametitle{杂项}
    \begin{itemize}
        \item 本来想用樱花实现一个更复杂的progress bar的,最后放弃了.
        \item 因为背景用的老图书馆的背景,使用一个带颜色的block显得多余,所以很多地方直接使用了 beamer 的默认配置.
        \item titlepage页\alert{默认没有}frametitle,所以如果需要顶部的背景色,需要使用$\backslash$frametitle\{$\backslash \backslash$\}.
        \item 樱花和老图书馆被封装为独立的模板,可以将其放在tikzpicture中进行修改.
        \item 有一些模板是可以重新设定的,可以见最后一页. \hyperlink{final}{\beamerbutton{Final}}
    \end{itemize}
\end{frame}

\end{document}
