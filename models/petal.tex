% petal.tex v1.2.3
% This model provides models for petals, flowers and other components.
% Please DO NOT compile this file directly.
% Author: Tony Xiang
% Licensed under the LaTeX Project Public License, v1.3c or later.

\providecommand{\drawpetalgeneral}[8]{
    % draw a general petal (not needed be called explicitly in most situations)
    % required parameter(s): #1 -- rotation of the petal
    %                        #2 -- outer color of the petal
    %                        #3 -- inner color of the petal
    %                        #4 -- kernel color of the petal    
    %                        #5 -- outwards angle of the petal
    %                        #6 -- inwards angle of the petal
    %                        #7 -- offset in x axis
    %                        #8 -- offset in y axis

    \filldraw[rotate=#1,color=#2] (0,0) [out=#5,in=#6]to (#7,#8) -- (0,#8-#7) -- (-#7,#8) [out=-180-#6,in=180-#5]to (0, 0) -- cycle;
    \fill[rotate=#1-6,#3,opacity=.65] (0,#8*0.25) circle (1.2pt);
    \fill[rotate=#1+6,#3,opacity=.65] (0,#8*0.25) circle (1.2pt);
    \fill[#4,opacity=.75] (0,0) circle (2.2pt);
}

\providecommand{\drawpetal}[2]{
    % draw a certain petal
    % required parameter(s): #1 -- rotation of the petal
    %                        #2 -- petal type

    \ifcase#2\relax
    \or\drawpetalgeneral{#1}{嫣红}{樱桃色}{樱草色}{53}{-25}{0.25}{3.75} % Type 1 -- pink petal of sakura
    \or\drawpetalgeneral{#1}{精白}{粉红}{樱草色}{53}{-25}{0.25}{3.75} % Type 2 -- white petal of sakura
    \fi    
}

\providecommand{\drawflower}[2]{
    % draw a flower
    % required parameter(s): #1 -- the number of petals
    %                        #2 -- petal type
    % e.g. for a pink sakura, call \drawflower{5}{1}

    \foreach \x in {1,...,#1}            
    {
        \drawpetal{-(\x - 1) / #1 * 360}{#2}
    }
}